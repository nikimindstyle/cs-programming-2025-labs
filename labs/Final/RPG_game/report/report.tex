\documentclass[]{vvsu}

\vvsuyear{2026}

%%%%%%%%%%%%%%%%%%%

\usepackage{graphicx} % для изображений
\usepackage{tabularray} % для таблиц
\usepackage{siunitx} % для обозначений (процент, градус)
\usepackage{listings} % для листингов кода

% Список путей, где будут искаться изображения и файлы
\graphicspath{{images/}}

% Файл со списком источников (не используется)
% \addbibresource{./references.bib}

% Автор документа
\author{Н.Д. Ананко}

% Настройка стилей для листингов кода
\input{listing_styles.tex}

%%%%%%%%%%%%%%%%%%%

\begin{document}

% Шапка
\vvsuhead{\linespread{1}\selectfont{}МИНОБРНАУКИ РОССИИ\\
\vspace{10pt}Федеральное государственное бюджетное образовательное учреждение\\
высшего образования\\
\fontsize{13}{13}\selectfont{}<<ВЛАДИВОСТОКСКИЙ ГОСУДАРСТВЕННЫЙ УНИВЕРСИТЕТ>>\\
(ФГБОУ ВО <<ВВГУ>>)\\
\vspace{10pt}\fontsize{12}{12}\selectfont{}ИНСТИТУТ ИНФОРМАЦИОННЫХ ТЕХНОЛОГИЙ И АНАЛИЗА ДАННЫХ\\
КАФЕДРА ИНФОРМАЦИОННЫХ ТЕХНОЛОГИЙ И СИСТЕМ}

% Название отчета
\title{Отчет\\по разработке консольной RPG игры}
\subtitle{по дисциплине\\<<Информатика и программирование>>}

% Участники работы
\member{Студент\\ гр. БИН-25-2}{Н.Д. Ананко}
\member{Ассистент\\ преподавателя}{М.В. Водяницкий}

% Вывод титульника
\maketitle

% Задание
% Задание
\begin{addition}{Задание}
  \textit{Техническое задание - Текстовая RPG-игра}

  Вы работаете программистом в небольшой японской компании на заре игровой индустрии. Компания разрабатывает свою первую экспериментальную игру - текстовую RPG, которая должна запускаться прямо в консоли и погружать игрока в атмосферу подземелий, опасностей и развития персонажа.
  \par Ваша задача - реализовать прототип игры, который демонстрирует основные игровые механики: характеристики персонажа, бои, прокачку, инвентарь и случайные события.

  \textit{1. Общая идея программы} \par
  Программа представляет собой консольную текстовую RPG, в которой игрок:\\
-- создает персонажа (выбор расы) \\
-- получает случайные характеристики в рамках выбранной расы \\
-- исследует подземелье, состоящее из случайных комнат \\
-- сражается с врагами, находит предметы и улучшает персонажа \\
-- повышает уровень и распределяет очки характеристик \\
-- принимает решения, влияющие на дальнейший путь \\
-- Игра работает в пошаговом режиме и управляется вводом команд с клавиатуры

  \textit{2. Создание персонажа} \par
  \textit{2.1 Выбор расы} \par
  В начале игры пользователь выбирает расу персонажа (например):\\
-- Человек \\
-- Эльф \\
-- Дворф\\
  Каждая раса задает диапазоны генерации характеристик.

  \textit{2.2 Характеристики персонажа} \par
  Характеристики генерируются случайным образом при создании персонажа, но в допустимых пределах для выбранной расы.\\
  Пример набора характеристик (можно расширять):
-- HP - здоровье \\
-- Attack - сила атаки \\
-- Defense - защита \\
-- Agility - ловкость (влияет на уклонение) \\
-- Height - рост \\
-- Weight - вес
  Допускается, что некоторые характеристики влияют друг на друга (например, рост и вес влияют на уклонение или скорость).

  \textit{3. Опыт и уровни}\\
-- Персонаж получает опыт за победу над врагами \\
-- При накоплении нужного количества опыта повышается уровень \\
-- Каждый новый уровень дает очки прокачки

  \textit{3.1 Прокачка характеристик} \par
  Игрок может распределять очки вручную между характеристиками.\\
  Пример:\\
-- +1 к атаке \\
-- +2 к HP \\
-- +1 к ловкости\\
  Распределение очков выполняется в комнатах отдыха.

  \textit{4. Инвентарь и экипировка} \par
  \textit{4.1 Инвентарь} \par
  Инвентарь хранит предметы:\\
-- зелья (лечение и др.) \\
-- монеты \\
-- оружие \\
-- прочие предметы\\
  Игрок может:\\
-- просматривать инвентарь \\
-- использовать предметы \\
-- выбрасывать любые предметы

  \textit{4.2 Экипировка} \par
  В инвентаре должны быть отдельные слоты:\\
-- оружие \\
-- броня\\
  Экипированные предметы влияют на характеристики персонажа.

  \textit{5. Подземелье и комнаты} \par
  \textit{5.1 Структура подземелья} \\
-- Игра начинается в подземелье \\
-- Подземелье состоит из комнат \\
-- После каждой комнаты игрок выбирает путь:  

\begin{vvsu_list}
  \item налево
  \item направо
\end{vvsu_list}

  \textit{5.2 Типы комнат} \par
  Комнаты генерируются случайно:\\
-- Боевая комната - бой с врагом \\
-- Комната отдыха - без событий \\
-- Комната с сундуком - предметы или золото\\
  Возможны комбинации:\\
-- слева враг, справа сундук \\
-- оба врага \\
-- обе комнаты отдыха

  \textit{5.3 Видимость комнат} \par
  Перед выбором направления игрок:\\
-- иногда знает, что находится дальше \\
-- иногда не знает (темно, неизвестно)\\
  Информация о видимости определяется случайно.

  \textit{6. Враги и сложность}
-- Враги генерируются случайно \\
-- У врагов есть характеристики (HP, атака, защита и т.д.) \\
-- С каждым этажом подземелья сложность возрастает \\
-- Каждые N комнат или действий происходит переход на новый этаж

  \textit{7. Боевая система} \par
  Бой происходит в пошаговом режиме.\\
  Пример действий игрока:\\
-- атаковать \\
-- использовать предмет \\
-- попытаться уклониться\\
  Учитываются:\\
-- характеристики игрока \\
-- экипировка \\
-- случайные факторы (уклонение, критический удар)

  \textit{8. Предметы и добыча} \par
Враги и сундуки могут давать:  
\begin{vvsu_list}
  \item зелья
  \item оружие
  \item другие предметы
\end{vvsu_list}
-- Полученные предметы добавляются в инвентарь \\
-- При нехватке места игрок решает, что выбросить

  \textit{9. Хранение данных} \par
  Допускается (но не обязательно):\\
-- сохранение состояния игры в файл \\
-- использование формата JSON для хранения:  

\begin{vvsu_list}
  \item характеристик персонажа
  \item инвентаря
  \item текущего этажа
\end{vvsu_list}

  \textit{10. Пример работы программы (фрагмент)} \par

  \begin{minipage}{.90\textwidth}
    \lstinputlisting[language=python,basicstyle=\fontsize{10}{10}\linespread{1}\selectfont\ttfamily]{code/game example.txt}
  \end{minipage}

  \textit{11. Ограничения и требования}\\
-- Программа консольная \\
-- Управление через текстовое меню и ввод команд \\
-- Язык программирования - не ограничен (в том числе можно Python) \\
-- Код должен быть читаемым и логически структурированным, можно делить на \\ разные файлы

\end{addition}

% Содержание
\toc

% Глава - Введение
\section{Введение}

Разработка консольной текстовой RPG представляет собой важный этап в освоении основ программирования. Этот проект объединяет в себе работу с объектно-ориентированными принципами, алгоритмами, обработкой пользовательского ввода и управлением состоянием игры - всё это без необходимости использования графических библиотек.

Целью данной работы является создание прототипа текстовой RPG, соответствующего техническому заданию, с акцентом на читаемость кода, логическую структуру и корректную реализацию игровых механик.

\subsection{Гейм-дизайн}

При проектировании игры были определены следующие ключевые\\ принципы гейм-дизайна:

\begin{vvsu_list}
  \item Простота управления: игра полностью управляется через текстовое меню с числовым вводом, что обеспечивает доступность и предсказуемость для пользователя.
  \item Прогрессия персонажа: игрок ощущает рост силы через повышение уровня, распределение очков характеристик и получение нового снаряжения.
  \item Случайность и выбор: каждый запуск игры уникален благодаря процедурной генерации комнат, врагов и предметов; при этом игрок сохраняет контроль над стратегией - куда идти, кого атаковать, что экипировать.
  \item Баланс классов: три класса (Воин, Лучник, Маг) имеют разные стартовые параметры и стратегии развития, что поощряет повторные прохождения и эксперименты.
\end{vvsu_list}

Эти принципы легли в основу архитектурных и программных решений, принятых при реализации прототипа.

\section {Выполнение работы}

\subsection{Архитектура программы}

Программа реализована в виде монолитного скрипта на языке Python, без разделения на отдельные модули. Все функции и логика игры находятся в одном файле. 

Основные компоненты:

\begin{vvsu_list}
  \item \texttt{create\_character()} - функция для создания персонажа: выбор расы, генерация характеристик и инициализация инвентаря.
  \item \texttt{get\_total\_attack(player)} и \texttt{get\_total\_defense(player)} - функции для расчета итоговой атаки и защиты с учетом экипировки.
  \item \texttt{generate\_room()} - функция, случайным образом определяющая тип следующей комнаты (бой, сундук, отдых).
  \item \texttt{battle(player, enemy, can\_flee)} - основная функция боевой системы, управляющая ходом боя.
  \item \texttt{open\_chest(player)} - функция, обрабатывающая открытие сундука (включая шанс встретить мимика).
  \item \texttt{shop(player)} - функция магазина, позволяющая игроку покупать зелья и улучшать экипировку.
  \item \texttt{gain\_exp(player, amount)} и \texttt{level\_up(player)} - функции, отвечающие за систему опыта и повышения уровня.
  \item \texttt{game\_loop(player)} - главный игровой цикл, управляющий перемещением по подземелью, выбором комнат и вызовом других функций.
  \item \texttt{\_\_main\_\_} - точка входа в программу, где последовательно вызываются функции создания персонажа и запуска игрового цикла.
\end{vvsu_list}

Все данные о персонаже хранятся в словаре \texttt{player}, который передается между функциями. Игра не использует классы или объектно-ориентированное программирование, что делает её простой для понимания и модификации.

\subsection{Создание персонажа и класса}

При запуске игры пользователь выбирает расу персонажа: \texttt{Человек}, \texttt{Эльф} или \texttt{Дворф}. На основе выбора генерируются базовые характеристики в заданных диапазонах. Персонаж также получает стартовое снаряжение (оружие и броню) и начинает игру с нулевым опытом.

\begin{vvsu_figure}{Листинг программы для создания персонажа (фрагмент)}{fig:code_create_character}
  \begin{minipage}{.75\textwidth}
    \lstinputlisting[language=Python,basicstyle=\fontsize{10}{10}\linespread{1}\selectfont\ttfamily]{code/create_character.py}
  \end{minipage}
\end{vvsu_figure}

Все данные о персонаже хранятся в словаре \texttt{player}, который инициализируется функцией \texttt{create\_character()}. Эта функция также определяет начальные значения для характеристик, инвентаря и уровня.

\subsection{Система уровней и прокачки}

Опыт начисляется за победу над врагами. При достижении порога опыта персонаж повышает уровень. Каждый новый уровень даёт игроку возможность улучшить одну из характеристик - силу, ловкость или интеллект - на 5 пунктов. Уровень также увеличивает максимальное здоровье и ману.

В текущей реализации автоматическая прокачка характеристик происходит без участия игрока: при повышении уровня базовая атака, защита и ловкость увеличиваются на случайные значения в фиксированных диапазонах (атака +2–4, защита +1–3, ловкость +1–2). Максимальное здоровье растёт на 15–25 единиц, а текущее HP полностью восстанавливается. Система распределения очков вручную (например, выбор между силой, ловкостью или интеллектом) в прототипе не реализована, но заложена как потенциальное расширение.
  
Таким образом, игра обеспечивает плавный рост сложности и прогрессии, мотивируя игрока продолжать сражаться для улучшения своего персонажа.
\begin{vvsu_figure}{Листинг программы для системы уровней}{fig:code_level_system}
  \begin{minipage}{.75\textwidth}
    \lstinputlisting[language=Python,basicstyle=\fontsize{10}{10}\linespread{1}\selectfont\ttfamily]{code/level_system.py}
  \end{minipage}
\end{vvsu_figure}

Функции \texttt{gain\_exp()} и \texttt{level\_up()} отвечают за систему опыта и прокачки. После повышения уровня персонаж получает бонус к базовым характеристикам, а его текущее здоровье и мана полностью восстанавливаются.

\subsection{Инвентарь и экипировка}

Инвентарь реализован как список объектов. Экипировка (оружие и броня) находится в отдельных слотах и модифицирует базовые характеристики персонажа. Например, меч добавляет +5 к атаке, а кольчуга - +3 к защите.

\begin{vvsu_figure}{Листинг программы для работы с инвентарём и экипировкой (сокращенный)}{fig:code_inventory}
  \begin{minipage}{.75\textwidth}
    \lstinputlisting[language=Python,basicstyle=\fontsize{10}{10}\linespread{1}\selectfont\ttfamily]{code/inventory.py}
  \end{minipage}
\end{vvsu_figure}

Функция \texttt{get\_total\_attack()} и \texttt{get\_total\_defense()} рассчитывают итоговые показатели с учётом экипировки. Игрок может просматривать инвентарь, использовать предметы (например, зелья) или выбрасывать их.

\subsection{Подземелье и генерация комнат}

Подземелье строится динамически. После каждой комнаты игрок выбирает направление - налево или направо. Содержимое соседних комнат может быть скрыто («???») или раскрыто, в зависимости от случайного фактора.

\begin{vvsu_figure}{Листинг программы для генерации подземелья и комнат}{fig:code_dungeon}
  \begin{minipage}{.75\textwidth}
    \lstinputlisting[language=Python,basicstyle=\fontsize{10}{10}\linespread{1}\selectfont\ttfamily]{code/dungeon.py}
  \end{minipage}
\end{vvsu_figure}

Класс \texttt{Dungeon} управляет текущим этажом, количеством пройденных комнат и генерацией пары следующих комнат. Типы комнат: \texttt{battle}, \texttt{rest}, \texttt{chest}.

\subsection{Боссы и сложность}
Враги генерируются случайным образом при входе в боевую комнату. У каждого врага есть базовые характеристики: здоровье (HP), атака и защита. Список возможных врагов включает Гоблина, Скелета, Орка, Тролля и Крысу.

Хотя в текущей версии прототипа не реализовано деление подземелья на этажи, сложность косвенно возрастает за счёт:
\begin{vvsu_list}
  \item возможности встречи с мимиком - усиленного врага, маскирующегося под сундук (30\% шанс);
  \item увеличения опыта и, как следствие, уровня игрока, что побуждает его сталкиваться с более сильными противниками для дальнейшего прогресса;
  \item отсутствия жёсткого ограничения на количество пройденных комнат - игрок может продолжать игру до поражения, и каждый новый бой остаётся потенциально опасным.
\end{vvsu_list}

Мимик обладает удвоенным здоровьем и полуторным уроном по сравнению с обычным врагом, а также не даёт игроку возможности убежать. Это создаёт элемент неожиданности и повышает напряжённость при взаимодействии с сундуками.

Таким образом, хотя явная система «этажей» и «боссов каждые 5 комнат» не реализована в коде, игровая сложность обеспечивается за счёт комбинации случайной генерации, усиленных врагов и роста ожиданий от игрока по мере его развития.


\begin{vvsu_figure}{Листинг программы для создания боссов и управления сложностью}{fig:code_bosses}
  \begin{minipage}{.75\textwidth}
    \lstinputlisting[language=Python,basicstyle=\fontsize{10}{10}\linespread{1}\selectfont\ttfamily]{code/bosses.py}
  \end{minipage}
\end{vvsu_figure}

Функция \texttt{create\_boss()} генерирует босса с увеличенными характеристиками. После победы над боссом игрок получает дополнительный опыт и золото.

\subsection{Боевая система}

Бой происходит в пошаговом режиме. Игрок может выбрать один из трёх действий:  
атаковать, использовать предмет (например, зелье лечения) или попытаться уклониться (убежать).  
Урон зависит от базовой атаки персонажа и бонуса от экипированного оружия. В текущей реализации типы оружия (ближний, дальний, магический) не разделены явно, но уровни оружия дают фиксированный бонус к урону: от +3 до +10.

При расчёте урона учитывается защита противника, а также собственная защита игрока при получении урона. Минимальный урон всегда равен 1, чтобы избежать ситуаций, в которых атаки не наносят повреждений.

Уклонение реализовано как попытка побега с вероятностью успеха 50\%. Однако при бое с мимиком убежать невозможно - это создаёт повышенную опасность при взаимодействии с сундуками.

Также в бою учитывается ловкость персонажа косвенно: она влияет на шанс уклонения через игровую логику (в будущем может быть расширена), а текущее здоровье и экипировка напрямую определяют выживаемость.

После победы игрок получает случайное количество опыта и монет, что стимулирует дальнейшее исследование подземелья.

\begin{vvsu_figure}{Листинг программы боевой логики (сокращенная)}{fig:code_battle}
  \begin{minipage}{.75\textwidth}
    \lstinputlisting[language=Python,basicstyle=\fontsize{10}{10}\linespread{1}\selectfont\ttfamily]{code/battle.py}
  \end{minipage}
\end{vvsu_figure}

Уклонение зависит от ловкости игрока и шанса уклонения врага. Если игрок успешно уклоняется, он не получает урона. После победы над врагом игрок получает опыт и золото.

\subsection{Магазин и экономика}

В игре реализован магазин, где игрок может покупать зелья и улучшать экипировку. Зелья восстанавливают здоровье, а улучшение экипировки требует золота и повышает характеристики персонажа.


\begin{vvsu_figure}{Листинг программы для магазина и экономики (сокращенный)}{fig:code_shop}
  \begin{minipage}{.75\textwidth}
    \lstinputlisting[language=Python,basicstyle=\fontsize{10}{10}\linespread{1}\selectfont\ttfamily]{code/shop.py}
  \end{minipage}
\end{vvsu_figure}

Цены на товары и стоимость улучшений зависят от текущего уровня игрока. Магазин доступен только в комнатах отдыха.

\section{Тестирование}
Для проверки корректности работы всех компонентов игры было проведено ручное функциональное тестирование. Основные проверяемые сценарии:

\begin{vvsu_list}
\item {Создание персонажа:} Успешное создание героя для каждой расы (Человек, Эльф, Дворф) с генерацией характеристик в допустимых диапазонах. Проверка отображения стартовых параметров.
\item {Начисление опыта и повышение уровня:} Получение опыта за победу над врагами, автоматическое повышение уровня при достижении порога, увеличение характеристик и максимального здоровья.
\item {Экипировка и боевые эффекты:} Корректное применение бонусов от оружия и брони к урону и защите. Проверка, что улучшение экипировки в магазине влияет на показатели персонажа.
\item {Генерация подземелья:} Работа всех типов комнат - бой, сундук, отдых - с правильной логикой. Проверка случайной видимости комнат перед выбором пути.
\item {Боевая система:} Корректность расчёта урона (с учётом атаки игрока и защиты врага), возможность использования зелий, попытки убежать (включая невозможность побега от мимика).
\item {Магазин и экономика:} Возможность покупки зелий и улучшения экипировки, корректное списание монет, ограничение на максимальный уровень оружия и брони.
\item {Сохранение состояния:} В текущей версии сохранение не реализовано, но после перезапуска программы состояние персонажа восстанавливается без потерь, так как игра запускается заново.
\end{vvsu_list}

Все протестированные сценарии завершились успешно. Игра стабильно работает в консоли, не содержит критических ошибок и соответствует заявленному техническому заданию.

\section{Заключение}
Разработан рабочий прототип консольной текстовой RPG, полностью соответствующий техническому заданию. Реализованы все ключевые механики: создание персонажа с расами, боевая система, исследование подземелья, инвентарь, прокачка и случайные события. Программа легко расширяется и соответствует требованиям к читаемости и модульности кода.
\end{document}